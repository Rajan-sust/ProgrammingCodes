\documentclass[14pt]{extarticle}
\usepackage[utf8]{inputenc}
\usepackage[a4paper, total={6in, 8in}]{geometry}
\usepackage{listings}
\usepackage{xcolor}
\lstset { %
    language=C++,
    backgroundcolor=\color{black!5}, % set backgroundcolor
    basicstyle=\footnotesize,% basic font setting
}
\title{Competitive Programming Precode}
\author{Rajan Saha Raju}
\date{\today}

\begin{document}
\maketitle
\newpage

\section{Input-Output}
\subsection{Printing Floating Point Number With Fixed Precision}

\begin{lstlisting}
printf("%.5f\n", x);
cout<<fixed<<setprecision(5)<<x<<endl;
\end{lstlisting}

\newpage

\section{Number Theory}
\subsection{Sieve of Erathosnes}

\begin{lstlisting}
const int N = 4000;
bitset<N + 5> mask;
vector<int> prime;

void sieve()
{
    for (int i = 3; i * i <= N; i += 2)
        if (!mask.test(i))
            for (int j = i * i; j <= N; j += (i << 1))
                mask.set(j);

    prime.emplace_back(2);

    for (int i = 3; i <= N; i += 2)
        if (!mask.test(i))
            prime.emplace_back(i);
}
\end{lstlisting}

\subsection{Mobius Function}
\begin{lstlisting}


const int N = 100000;
vector<int> u(N + 1, 1), factor(N + 1);

void getMobius()
{

    for (int i = 2; i <= N; i++) {

        if (factor[i] == 0) {

            for (int j = i; j <= N; j += i) {

                int temp = j, cnt = 0;
                while ((temp % i) == 0) {
                    temp /= i;
                    cnt++;
                }
                if (cnt >= 2) {
                    u[j] = 0;
                }
                factor[j]++;
            }
        }
    }

    for (int i = 2; i <= N; i++) {

        if (u[i] == 1)
            u[i] = (factor[i] & 1) ? -1 : 1;
    }

    return;
}

\end{lstlisting}

\subsection{Finding Divisor}

\begin{lstlisting}
void getDivisor(int n,vector<int>&v) {

	/* v for storing all divisor and prime has been 
	already calculated using sieve. */
	v.emplace_back(1);
	for(int i=0;prime[i]*prime[i]<=n;i++) {
		int mul = 1;
                int tem = v.size();

		while(n%prime[i]==0) {
			n/=prime[i];
			mul*=prime[i];
			for(int j=0;j<tem;j++){
				v.emplace_back(mul*v[j]);
			}
		}
	}

	if(n!=1){
		int tem = v.size();
		for(int j=0;j<tem;j++){
			v.emplace_back(v[j]*n);
		}
	}
}
\end{lstlisting}
\end{document}














